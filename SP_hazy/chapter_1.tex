%   Filename    : chapter_1.tex 
\chapter{Introduction}
\label{sec:researchdesc}    %labels help you reference sections of your document

\section{Overview of the Current State of Technology}
\label{sec:overview}
The Department of Environment and Natural Resources (DENR) expressed in their website that monitoring air quality is essential in reducing air pollution and they plan to protect the environment and public health by strengthening their air quality monitoring systems \cite{DENR2020}.

An example of air quality monitoring system that the DENR uses is the Differential Optical Absorption Spectroscopy abbreviated as DOAS. \cite{DENR_ND}. DOAS captures light that passes through a medium (the medium in this case is the atmosphere) to measure different wavelengths that were absorbed by different gasses. This method can accurately measure trace gasses absorption and it is simpler and less expensive to operate. DOAS, however, is greatly affected by turbulence in the atmosphere \cite{PlattEtAl2008}. DENR also has Particulate matter stations that records PM 2.5 and PM 10. \cite{DENR_ND}

DOAS equipments need frequent maintenance to be able to operate normally. In a news report by \cite{enano_subingsubing_2019} regarding air pollution at EDSA it was stated that the maintenance of these equipements requires “hundreds of thousands of pesos”.

Currently, The way to access the data from the AQMS stations is through the website (https://air.emb.gov.ph/ambient-air-quality-monitoring/) and the play store application.

  
%--- the following example shows how to include a figure in PNG format
\begin{figure}[t]                %-- use [t] to place figure at top, [b] to place at the bottom, [h] for here
   \centering                    %-- use this to center the figure
   \includegraphics[width=120mm,scale=1.5]{PAQIApp.jpg}      %-- include image file named as "disneychart.png" 
   \caption{Screen capture of the mobile application of the Philippines Air Quality Index. Photo taken from https://air.emb.gov.ph/ambient-air-quality-monitoring/ 
}
    \label{fig:PAQIApp}
\end{figure}


In the mobile application disclaimer it is stated that the system audits are irregular and all data are subject to quality Assurance and quality control. This means that the end user may not get the accurate data that they expect when using the application. Moreover, monitoring stations are not online 24/7 which makes the data much less available.


\section{Problem Statement}
	Air pollution has become a global problem over the years. As stated by \cite{Akimoto2004}, the availability of the CO2 concentrations on the Measurement of Air Pollution from Satellite (MAPS) instrument in 1981 shows high concentrations of the greenhouse gas over tropical Asia, Africa, and South America. Not only does this date provide evidence that this has become an international issue, it also shows how fossil fuel combustion can have an impact on air quality. 

The Philippines, a country located in tropical Asia, is not devoid of these issues. An article by \cite{abano_2019} states that a 2018 study by the World Health Organization reports the Philippines has ranked third among the countries with air pollution related deaths. These deaths are tied to harmful particles entering a person’s lungs, which can lead to multiple different ailments and diseases such as: heart disease, lung cancer, and respiratory infections, to name a few.

Air pollution can come from different sources, whether it be from stationary constructs like factories or mobile sources such as cars. \cite{EMB_2015}  An air quality status report by the Department of Environment and Natural Resources (2015) shows that 65\% of the air pollutants come from these mobile sources. This worsened as the EMB’s official site \cite{EMB_2018} states that based on the Emissions Inventory of 2018, the pollutant contribution of mobile sources has increased to  74\%. In places where traffic is congested could be a huge contributing factor to vehicular emissions. \cite{vergel_yai2000} states that the congestion in the roads of Metro Manila contributes to the worsening air quality, especially in the vicinity of the road environment.

In the country’s attempt to mitigate the atmospheric issues, the Philippine Clean Air Act of 1999 (Republic Act No. 8749) was passed. \cite{FAO} It entails the resolution of creating a national program of air pollution management, mainly focusing on pollution prevention. Two decades later and the country still sees increasing pollutants in the air and does not show signs of the improvement that was planned.

In addition to creating air pollution management programs, the Philippine Clean Air Act also aimed to utilize mass media communication in order to create awareness and active participation in air quality planning and monitoring. With the aforementioned said, a system that could satisfy that goal can be created with the newer technologies that were not present in the decades prior. Thus creating Hazy, a system that can gather information from one of the biggest contributors of air pollution and provide an estimate to emissions produced, can provide solutions to these problems. 


\section{Research Objectives}
\label{sec:researchobjectives}

\subsection{General Objective}
\label{sec:generalobjective}


The general objective of the study is to develop an application that calculates an area’s average amount of Carbon Dioxide emission through the use of a vehicle detection system. This system will identify the vehicles from a live camera feed, along with their makes and models. The system will integrate the database of CO2 emissions the vehicles, which then the average will be calculated to be displayed on the application for the user to see.



\subsection{Specific Objectives}
\label{sec:specificobjectives}

%
%  \begin{comment} ... \end{comment} is used for multiple lines of comment
%
This study specifically aims to:


\begin{comment}
% IPR acknowledgement: the following sentences and examples are from Ethel Ong's slides 
%     on Research Objectives
How to formulate your research objectives:
1. Identify what research steps do you need to perform to achieve your general objective.
2. Identify the questions that must be answered for you to achieve your general objective.
    Thereafter, convert these questions into action statements

Example #1:

Research Question:
  What are the general features of a web-based learning environment?

Specific Objective:
   To review existing web-based learning environment that teaches language learning for children


Example #2:

Research Question:
   How will you represent commonsense knowledge for use by computer systems?

Specific Objective:
   To identify knowledge representation approaches used by existing story generation systems

Example #3:
Research Question:
   What types of storytelling knowledge are needed to generate stories?

Specific Objective:
    To identify the different types of storytelling knowledge used in generating stories

Example #4:
Research Question:
    What machine learning approaches will you utilize?

Specific Objective:
    To determine existing machine learning algorithms [that can be used in training the computer system to detect cyberbullying cases] 

Example #5: Research Question:
    How will your research output be evaluated?

Specific Objective:
    To define evaluation metrics for validating the accuracy of the translation

\end{comment}

%
%  The following are example specific objectives; replace them with your own 
%

\begin{enumerate}
   
   \item To study YOLOv5 and its components to develop the application
   \item To implement real-time vehicle detection that can be used on a live camera feed
   \item To integrate a database of vehicular CO2 emissions so the application can calculate the overall emission of an area based on the vehicles' average emission.

\end{enumerate}


\section{Scope and Limitations of the Research}
\label{sec:scopelimitations}
This application mainly focuses on the air pollution caused by traffic in the Philippines, where the researchers reside as of writing the paper. Thus, it will only be set up and used on the vehicles that travel within the country. The researchers will be limited to the amount of vehicles that are present within the database that will be used for the application. Any other information that is not present in the database will not be included in the application. Among the components of air pollutants, this study is limited to using Carbon Dioxide as the emission to be identified from the vehicles. Furthermore, This study will be utilizing the YOLOv5 object identification framework  and is thus limited to the features of that version. Any other features and upgrades that are present in future versions of the framework will not be included in the study.


\begin{comment}

%
% IPR acknowledgement: the sentences inside this comment are from Ethel Ong's slides on Scope and Limitations of the Research
%
Generally, one paragraph should be allotted for each of your research objectives.

Each paragraph contains a brief overview of the concept/theory and the purpose of doing the associated objective.

Each paragraph also includes a description of the scope/limitation of your study.

* Please refer to the slides for examples.

\end{comment}


\section{Significance of the Research}
\label{sec:significance}

The main objective of this study is to create an application that helps its users identify the pollution level of a traffic congested area through a live video feed over the road. Furthermore, this benefits users such as those who travel frequently by the road whether it be: commuters, joggers, or workers, in which they have the opportunity to check said road’s emission levels. In such a case where levels are higher than expected, users can take premeditated measures to avoid the intake of greenhouse gasses. Moreover, this study will be of significance to individuals to immuno-compromised individuals that should steer clear of areas that can cause a negative impact on their health. 

For the environment sector, this study can help contribute to the air pollution awareness in the country, in which such data can be utilized when creating future plans and protocols to combat the rising concern for the country’s air quality.

Lastly, as interest in the computer vision field of vehicle identification and recognition systems increases, this study can contribute to future research on said field. The study can be of help to future researchers on the topics of: computer vision and tracking vehicular greenhouse gasses emissions.



\begin{comment}
If applicable, describe possible commercialization and/or innovation in your research.
\end{comment}


