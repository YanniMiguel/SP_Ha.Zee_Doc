%   Filename    : chapter_2.tex 
\chapter{Review of Related Literature}
\label{sec:relatedlit}

This chapter discusses the features, capabilities, and limitations of existing research, algorithms, or software  that are related/similar to the Special Problem.

 The reviewed works and software must be arranged either in chronological order, or by area (from general to specific).  
Observe a consistent format when presenting each of the reviewed works. 
This must be selected in consultation with the  adviser.

\textcolor{red}{DO NOT FORGET to cite your references.}


\begin{comment}
%
% IPR acknowledgement: the contents withis this comment are from Ethel Ong's slides on RRL.
%
Guide on Writing your RRL chapter
 
1. Identify the keywords with respect to your research
      One keyword = One document section
                Examples: 2.1 Story Generation Systems
			 2.2 Knowledge Representation

2.  Find references using these keywords

3.  For each of the references that you find,
        Check: Is it relevant to your research?
        Use their references to find more relevant works.

4. Identify a set of criteria for comparison.
       It will serve as a guide to help you focus on what to look for

5. Write a summary focusing on -
       What: A short description of the work
       How: A summary of the approach it utilized
       Findings: If applicable, provide the results
        Why: Relevance to your work

6. At the end of each section,  show a Table of Comparison of the related works 
   and your proposed project/system

\end{comment}

\section{Air Quality Monitoring Systems}
Air quality monitoring systems are systems that collects data to record and or or analyze atmospheric emission levels. There are various systems for air quality monitoring. Zoogman et al. (2016) showcased in a journal the use of satellite imagery for large-scale air quality monitoring. They call this instrument TEMPO (Tropospheric Emissions: Monitoring of Pollution). It is an instrument that collects data on tropospheric emissions such as NO2, SO2, H2CO, Methane, etc from a satellite in a geostationary orbit. This system is wide-range and precise,  however, access to the equipment is limited.  A more accessible air monitoring system was made byy Zheng et al. (2016) using several sensors. This system makes use of LPWA (low power wide area) to give it a wider coverage compared to the IoT (Internet-of-Things ) and the air quality data can be accessed through a mobile application. These systems make use of dedicated sensors to collect emission data whereas this project will make use of computer vision and machine learning.


\section{Air Pollution from Vehicles}
The Philippines currently has a problem with air pollution. According to Tentengco et al. (2022), the Philippines’ PM2·5 concentrations in urban areas exceed the WHO guideline value. They further state that the Philippines’ PM2·5 levels reaches 58·4 ug/m3 in traffic sites of Metro Manila during the dry season. Though there could be different sources of air pollution, 65 percent of the air pollutants come from mobile sources such as: cars, motorcycles, trucks, and buses (DENR Philippines, 2018).

	Furthermore, CO2, a component of greenhouse gasses, totaled  to “30 million tons and 56 thousand tons of particulate matter” (Herbert \& Sudhir, 2009) in the Philippines and the transport sector contributed to 38 percent of fuel combustion back in 2000. The authors have noted that the motorized vehicle count would double by 2020. The increase of motorized vehicles also means an increase in its air pollution contribution. 

\section{Vehicle Tracking}
Vehicle tracking can be used for identifying the information of the vehicle, such as its brand and model. A recent paper by Saravi and Edirisinghe (2019) presents a Vehicle Make and Model Recognition (VMMR) System that accepts a video feed and returns the make and model of the vehicles detected on the feed. This system tracks the vehicle’s license plate followed by selecting the region of interest above the plate–the vehicle. As the vehicle moves along the camera, its license plate is also detected across multiple frames while motion segmentation was used to keep track of the static area above the license plate.

	Aside from finding the vehicle’s plate number, another process to detect and track the vehicle would be through background subtraction. Background subtraction, according to Huang BJ. et al.(2017, as cited in Manzoor, 2018), is used to extract the moving objects and then filter the unwanted images through image processing tools. 


\section{Application of YOLO}
From the YOLOv5 Documentation (https://docs.ultralytics.com/) website:  “YOLO, an acronym for 'You only look once', is an object detection algorithm that divides images into a grid system. Each cell in the grid is responsible for detecting objects within itself.” This is a useful tool for identifying objects in an image or video.

One application of this algorithm was done by Yan et al. (2021) for an apply picking robot. YOLOv5 was used to identify apples, however the algorithm cannot detect apples that are safe to pick and those that are not. This may cause the picking arm of the robot to break if it tries to grasp an apple that is occluded by a solid object. They solved this problem by improving on the modules used for the algorithm. This is not a problem for this project as it only counts the number of vehicles without interacting with them. 

In a study done by Zhou et al. (2021), they applied YOLOv5 algorithm to detect safety helmets on workers. The algorithm had an average detection speed of 110 fps in real-time. With a 94.7\% effectiveness (The model was trained and tested  using 6045 data sets) the algorithm proved to be viable for real-time detection.









