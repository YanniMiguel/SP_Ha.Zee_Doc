
%   Filename    : abstract.tex 
\begin{abstract}
	
	Air pollution is a global problem and the Philippines ranked third among countries with deaths relating to air pollution. Mobile sources are responsible for 65\% of the pollutants in the atmosphere and for two decades the country has tried to mitigate these atmospheric issues but shows no improvement. Air quality monitoring is important for mitigating air pollution in the Philippines. However, The country is doing a poor job of maintaining air quality monitoring devices to keep them operational. Moreover, it is expensive to maintain these tools, and access to the data is limited. This project aims to utilize new technologies to develop an alternative air quality monitoring system to help bring a solution to this problem. this project will utilize an object detection algorithm, YOLOv5 to be trained to identify and count the number of vehicles on the road to estimate the amount of fine particulate matter ($PM_{2.5}$) present in an area.
	
	
	%  Do not put citations or quotes in the abract.
	
	
	\begin{flushleft}
		\begin{tabular}{lp{4.25in}}
			\hspace{-0.5em}\textbf{Keywords:}\hspace{0.25em} & Machine Learning, Artificial Intelligence, Object Detection\\
		\end{tabular}
	\end{flushleft}
\end{abstract}